The finite-volume QCD spectrum in the $I=\frac{1}{2}$, $S=-1$, parity-even, zero-momentum sector containing the $\kappa$ meson and the $I=1$, $S=0$, parity-even, $G$-parity-odd, zero-momentum sector containing the $a_0(980)$ meson was studied with the inclusion of tetraquark operators using lattice QCD. This work is the most comprehensive lattice study of tetraquark operators to date in the $\kappa$ and $a_0(980)$ sectors, and is the first to study tetraquarks in the $\kappa$ sector neglecting no disconnected diagrams. The spectrum of excited $\Sigma$ baryons in the $I=1$, $S=-1$, parity-even and parity-odd sectors was also studied using large bases of single- and two-hadron operators. This is the first study of the excited $\Sigma$ baryon spectrum to include two-hadron operators. All calculations were performed using 412 gauge field configurations using clover-improved Wilson fermions on a $32^3\times 256$ anisotropic lattice with $m_\pi \approx 230$ MeV, and quark propagation was treated using the Stochastic LapH method.

We found that including a tetraquark operator in the $\kappa$ channel produced an additional state in our spectrum determination which was not present without the tetraquark operator. This additional state is in the range where we expect to find the experimental $\kappa$, and is within error of a qualitative determination of the $\kappa$ mass on the same lattice from Ref.~\cite{Brett:2018jqw}. This significant result suggests that there is a state in the finite-volume lattice spectrum that shares quantum numbers with the $\kappa$ resonance, and that has tetraquark content. In the $a_0(980)$ channel, we also found that including a tetraquark operator produced an additional state in our spectrum determination which was not present without the tetraquark operator. Additionally, we found that our determination of the other energies in the spectrum was dramatically affected by the inclusion of the tetraquark operator. These significant results underscore the importance of including tetraquark operators in studying the $\kappa$ and $a_0(980)$ resonances. For both the $\kappa$ and $a_0(980)$ sectors, a more detailed examination of the role of the tetraquark operators will require the L\"uscher method with increased statistics and tetraquark operators of nonzero momenta.

In the $\Sigma$ baryon sector, we successfully extracted the spectrum of excited $\Sigma$ baryons using a large basis of single- and two-hadron operators, and found qualitative agreement with experiment. We mostly found qualitative agreement to a previous study using a smaller lattice, a heavier pion mass, and no two-hadron operators. Our results are significant and illustrate the importance of including two-hadron operators when extracting an excited hadron spectrum.

Given that one of the main limiting factors on the computational feasibility of lattice QCD calculations is the light quark mass, increased computational power and improved techniques may allow for similar calculations done here to be performed at the physical pion mass in the future. Such calculations would allow for better comparison of our results to experiment, and offer more insight into the pion-mass-dependence of the QCD spectrum.